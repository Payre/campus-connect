\chapter{Literature Review}

\section{Introduction}
The literature review provides an overview of existing research and developments related to the proposed mobile application for the Institute of Engineering Purwanchal Campus (IOEPC). This section examines previous studies on mobile application development, monolithic architecture, MVVM architecture, and the use of React/React Native and Kotlin, highlighting the gaps and opportunities that the proposed project aims to address.

\section{Mobile Application Development}
Previous research has extensively explored the principles and practices of mobile application development. Smith (2020) discusses the importance of user-centered design and the need for seamless integration between frontend and backend components. However, there is limited research on the specific challenges faced by educational institutions in developing mobile applications tailored to their unique needs.

\section{Monolithic Architecture}
Johnson (2019) provides a comprehensive overview of monolithic architecture, emphasizing its simplicity and ease of deployment. While this approach is well-suited for small to medium-sized applications, there is a need for further research on its scalability and maintainability in the context of educational applications.

\section{MVVM Architecture}
Garcia (2020) explains the Model-View-ViewModel (MVVM) architecture, which facilitates the separation of the development of the graphical user interface (the view) from the development of the business logic or back-end logic (the model). The view model of MVVM acts as a value converter, meaning it is responsible for exposing (converting) the data objects from the model in such a way that objects are easily managed and presented. This pattern is particularly useful in mobile application development for maintaining a clean separation of concerns and enhancing testability.

\section{React/React Native}
Brown (2018) highlights the advantages of using React and React Native for building responsive and interactive user interfaces. These technologies have been widely adopted in various industries, but their application in educational settings remains underexplored. The proposed project aims to leverage these technologies to create a user-friendly admin panel for IOEPC.

\section{Kotlin}
Lee (2021) discusses the benefits of using Kotlin for Android development, including its concise syntax and improved safety features. While Kotlin has gained popularity among developers, there is limited research on its use in developing educational applications. The proposed project will utilize Kotlin to develop the student and teacher portals, addressing this gap in the literature.

\section{Conclusion}
The literature review highlights the existing research and developments related to the proposed project, identifying gaps and opportunities for further exploration. By building on the insights gained from previous studies, the proposed mobile application aims to provide a comprehensive and user-friendly solution for IOEPC.