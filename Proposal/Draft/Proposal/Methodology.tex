\chapter{Methodology}

\section{Overview}
This section outlines the systematic approach that will be employed to develop the proposed mobile application for the IOEPC. The methodology includes phases of development, tools and technologies used, and strategies for testing and deployment. The application will be developed using a monolithic architecture, with the frontend aspects such as the admin panel implemented in React/React Native, and the student and teacher portals designed in Kotlin.

\section{Requirements Gathering}
\begin{itemize}
    \item \textbf{Interviews and Surveys}: Conduct interviews and surveys with students, faculty, and administrative staff to gather requirements and understand the current challenges.
    \item \textbf{Document Analysis}: Review existing documents, such as academic schedules, notice formats, and library management processes.
\end{itemize}

\section{System Design}
\begin{itemize}
    \item \textbf{Architecture Design}: Design a monolithic architecture to ensure simplicity and ease of deployment.
    \item \textbf{Database Design}: Develop a relational database schema to store user data, academic schedules, notices, and other relevant information.
    \item \textbf{UI/UX Design}: Create wireframes and prototypes for the mobile application to ensure a user-friendly interface.
\end{itemize}

\section{Development}
\begin{itemize}
    \item \textbf{Technology Stack}: Use React/React Native for the admin panel and Kotlin for the student and teacher portals. The backend will be developed using Node.js and Express.js.
    \item \textbf{Module Development}: Develop individual modules for different functionalities, such as notice management, academic schedules, library management, and user authentication.
    \item \textbf{Integration}: Integrate all modules to ensure seamless communication between different components of the application.
\end{itemize}

\section{Testing}
\begin{itemize}
    \item \textbf{Unit Testing}: Perform unit testing for individual modules to ensure they function correctly.
    \item \textbf{Integration Testing}: Conduct integration testing to verify that different modules work together as expected.
    \item \textbf{User Acceptance Testing (UAT)}: Involve a group of students and faculty in testing the application to gather feedback and make necessary improvements.
\end{itemize}

\section{Deployment}
\begin{itemize}
    \item \textbf{Beta Release}: Deploy a beta version of the application to a limited group of users for initial feedback.
    \item \textbf{Full Deployment}: Roll out the final version of the application to all students and faculty after incorporating feedback from the beta release.
\end{itemize}

\section{Maintenance and Updates}
\begin{itemize}
    \item \textbf{Monitoring}: Continuously monitor the application for any issues or bugs.
    \item \textbf{Regular Updates}: Provide regular updates to add new features and improve existing functionalities based on user feedback.
\end{itemize}

\section{Documentation}
\begin{itemize}
    \item \textbf{User Manual}: Create a comprehensive user manual to guide users on how to use the application.
    \item \textbf{Technical Documentation}: Provide detailed technical documentation for future developers to understand the system architecture and codebase.
\end{itemize}