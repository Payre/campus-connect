\chapter{TECHNOLOGY USED}

\section{Front-End}

\subsection{React}
React, an open-source project developed by Meta, is a JavaScript library that permits developers to create an interactive user interface for web applications. It helps in simplifying the building of complex user interfaces. The major work of React involves breaking down the UI into reusable components that are then utilized to construct the more complex UI elements. It supports code reusability and makes the use of state changes easier within the application. In React, there is the virtual DOM whose implementation helps to improve the time and performance for rendering the content in DOM \cite{react2025}.

\subsection{TypeScript}
To add the optional static typing to the language, a superset of JavaScript is used which is known as TypeScript. It is better known to improve code quality and maintainability by catching the bugs at the time of compiling. It comprises several features like auto-completion, code navigation, and helping to prevent common errors during programming. With the help of TypeScript, the developers write more maintainable and scalable code creating a better user experience for their applications \cite{typescript2025}.

\subsection{JavaScript}
JavaScript is a powerful and widely-used programming language that enables developers to create interactive and dynamic web applications. It is primarily used for client-side scripting, allowing web pages to respond to user actions without requiring a page reload. JavaScript can manipulate HTML and CSS to update content, control multimedia, animate images, and much more. Its versatility and ease of integration with other web technologies make it an essential tool for modern web development. Additionally, JavaScript supports asynchronous programming, which enhances the performance and responsiveness of web applications by allowing tasks to run in the background \cite{javascript2025}.

\subsection{Kotlin}
Kotlin, developed by JetBrains, is a statically typed programming language that is fully interoperable with Java. It features a concise syntax, reducing boilerplate code, and introduces null safety to prevent common null pointer exceptions. With smart casts, extension functions, and built-in support for asynchronous programming via coroutines, Kotlin provides a modern and efficient coding experience. It is widely adopted for Android development due to its compatibility with Java and its advanced features \cite{kotlin2025}.

\section{Back-End}

\subsection{Node.js}
Node.js is a popular and powerful backend framework that enables developers to build fast, scalable, and event-driven applications using JavaScript. It’s known for its efficient I/O operations, non-blocking I/O model, and a vast library of modules and packages via its package manager, npm. Node.js is also flexible and easy to learn, thanks to its JavaScript syntax and supportive community. With Node.js, developers can build robust and high-performance backend systems for web and mobile applications alike \cite{casciaro2020nodejs}.

\subsection{PostgreSQL}
PostgreSQL is widely used by organizations of all sizes for storing and managing their data. It is a powerful, free, and open-source database management system known for its ability to handle complex transactions and robust security features. Developers can extend and customize PostgreSQL to match their specific needs. It is a reliable database solution that is beneficial for any organization intending to store and manage their data efficiently.