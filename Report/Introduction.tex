\pagenumbering{arabic}
\chapter{INTRODUCTION}

\section{Background}
The Institute of Engineering Purwachal Campus (IOEPC) is an education institution in Nepal, offering a range of undergraduate and graduate programs in various engineering disciplines. Despite advancements in technology, many administrative and academic processes at campus remain manual and offline, leading to inefficiencies and delays. Technical processes are still being handled through paper-based methods, which are prone to errors and time-consuming. There is no direct connection between teachers and students, leading to delayed communication due to the dependency on class representatives. To address these issues, we propose the development of a mobile application that integrates all IOE services, providing a centralized and efficient platform for users. This application aims to streamline processes such as attendance management and assessment marks, making it easier for teachers to conduct daily activities and ensuring easy integration of existing offline systems and flexibility for future enhancements.

\section{Gap Identification}
Despite the availability of various educational management systems, there is a lack of tailored solutions that cater specifically to the needs of our college. Our applications address this gap by providing customized functionalities that align with the requirements of students, teachers, and administrators.

\section{Motivation}
The motivation behind this project is to simplify and automate the routine tasks of students, teachers, and administrators. By leveraging technology, we aim to reduce manual efforts, minimize errors, and improve the overall user experience. Our team, comprising students of IOEPC, saw an opportunity to contribute to the institution by developing a solution that benefits both students and the institute.

\section{Objectives}
\begin{itemize}
  \item Develop a user-friendly application for teachers to manage attendance, internal marks, notes, and notices.
  \item Create a student application that allows students to view their attendance, internal marks, notes, and notices.
  \item Implement an admin interface to manage user accounts for students and teachers.
  \item Improve online access to the educational system by using a monolithic architecture.
\end{itemize}

\section{Scope}
The proposed applications can be adopted by educational institutions to streamline their academic and administrative processes. The teacher application allows teachers to manage attendance, internal marks, notes, and notices efficiently. The student application provides students with easy access to their academic information. The admin interface enables administrators to manage user accounts for students and teachers, ensuring a smooth and secure user experience. The application will serve as an external interface, facilitating activities such as result publication and online attendance, and integrating existing systems to better serve stakeholders.

\subsection{Academic Study/Research}
Academic research is a crucial scope of our project. By sharing our application implementation with other researchers, they can benefit from the study of educational management systems, optimization techniques, and user experience improvements. Our customizable applications with an easy-to-use interface have an important scope in academia.

\subsection{Comparison of Scope with Existing Systems}
There are various educational management systems available, but there are several key differences between these technologies and our applications. Existing systems may not be tailored to the specific needs of our college, whereas our applications provide customized functionalities that align with the requirements of students, teachers, and administrators. Our applications support real-time data access and management.